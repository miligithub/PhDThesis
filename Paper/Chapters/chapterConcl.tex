%========================================
%                            Chapter                            
%======================================== 
\chapter{Conclusion}\label{chap:concl}

In this thesis, we studied three privacy and security problems on smartphones, where {\spp} is about  


. First, we focused on privacy issues in crowdsensing, which is about the safeguarding of users. We identified the privacy problems in task allocation and incentive mechanisms raised by inference attack. To preserve users’ location privacy during task allocation, we proposed two task allocation algorithms which maximize the number of assigned tasks while providing personalized loca- tion privacy protection against location-inference attack in Chapter 3. To protect users’ bid privacy in incentive mechanisms, we designed two frameworks for privacy-preserving auction- based incentive mechanisms in Chapter 4. Next, we shifted our focus to security issues in crowdsensing, which is about the safeguarding of the system. Specifically, we identified the security problems in incentive mechanisms and truth discovery raised by Sybil attack. To deter users conducting Sybil attack in incentive mechanisms, we designed Sybil-proof incen- tive mechanisms for both offline and online scenarios in Chapter 5. To diminish the impact of Sybil attack on the aggregated data, we proposed a Sybil-resistant truth discovery frame- work in Chapter 6. The major novelty and contribution of this thesis lie in two parts. The first part is identifying privacy and security issues in crowdsensing. Although many works aim to improve the performance of crowdsensing systems, the potential privacy and security issues might undermine crowdsensing. In this thesis, we identified privacy and security issues in there parts in crowdsensing: task allocation, incentive mechanisms, and truth discovery. Pointing out these issues provides new aspects to evaluate the vulnerability of crowdsensing. The second part is solving the identified privacy and security issues. Solving privacy and security issues is much harder than identifying them. Sometimes it is impossible to fully eliminate the impact of an attack. For each of the aforementioned issues, we proposed cor-
128responding solutions. These solutions can be treated as a guideline for future solutions to the corresponding privacy and security issues.
%========================================
%                            Section                             
%======================================== 	 

