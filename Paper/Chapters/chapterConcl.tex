%========================================
%                            Chapter                            
%======================================== 
\chapter{Conclusion}\label{chap:concl}
In this thesis, we studied three privacy and security problems on smartphones.
We first uncovered a new stealth attack named the {\attackName} attack that eavesdrops on smartphones' built-in speakers by the intra-device motion sensors. The attack is implemented in {\spp} system utilizing speaker-independent machine learning, which makes the attack more dangerous and harmful. We also provided hardware-based defenses and software-based defenses for this attack, but this attack is still a threat for smartphone users and requires user awareness. We then introduced two different authentication methods. The {\uu} system enables users to unlock the smartphone with gestures in the air.  It is a good alternative to existing authentication methods. Moreover, the same technique can be used for smartphone control, which allows users to unlock and control the phone without touching the screen.  The {\mv} system is a patch for the current voice authentication mechanism. It defends smartphones against various voice-spoofing attacks, especially the replay attack. The three systems either propose new S\&P problems on smartphones, or solve existing S\&P problems with novel approaches. However, we have also noticed the limitations of the current work and will continue to improve it in future work.

