%========================================
%                           Packages                             
%========================================
\usepackage{lipsum} % dummy text
\usepackage{graphicx} % for images
\usepackage{natbib} % for Chicago style references
% hyperref is a package that frequently requires compiling twice 
\PassOptionsToPackage{hyphens}{url}
%\usepackage{hyperref}
\usepackage{hyperref} % for \url and links 
\usepackage{pdflscape} % for landscape mode page
\usepackage{everypage} % for landscape mode page numbering 
\usepackage{afterpage} 
\usepackage{longtable} % for really long tables
\usepackage{xcolor} % blue


\urlstyle{same}
\hypersetup{hidelinks}
% Place all chapter tex files in ``Chapters'' folder
% Place all references bib files in ``References'' folder

% Place all figures in ``Figures'' folder
\graphicspath{{Figures/}{Figures/SpyPhone/}{Figures/MoVo/}{Figures/UltraUnlock/}{/Users/mili/Desktop/20200511Desktop/EavesProcessFinal/}{/Users/mili/Documents/UltraUnlock/}} 
% Or define several directories to be searched for figures.
% \graphicspath{{Figures/}{../figures/}{C:/Users/me/Documents/project/figures/}}

\DeclareUnicodeCharacter{FF09}{FIX ME!!!!}
%========================================
%                           Abbreviation                            
%========================================

\newcommand{\uu}{Ultra-Unlock}
\newcommand{\spp}{Spy-Phone}
\newcommand{\mv}{MoVo}

%========================================
%                           SpyPhone                       
%========================================
%\newcommand{\shortName}{Smartphone or Spy-phone? }
\newcommand{\shortName}{Spy-Phone}
\newcommand{\attackName}{Man-in-the-Phone}
\newcommand{\systemName}{\mbox{Spy-Phone}}
\newcommand{\systemNameCap}{\mbox{SPY-PHONE}}
\newcommand{\longName}{\emph{\shortName}: Eavesdropping on \\Smartphone Speakers with Motion Sensors}

\usepackage{booktabs}
%\usepackage{pifont} %\cmark \xmark

\usepackage{makecell} %\makecell
\usepackage[ruled,vlined]{algorithm2e}

\usepackage{amsmath}
\DeclareMathOperator*{\argmin}{argmin}  

%\usepackage[font=small,labelfont=bf]{caption} \normalsize

\usepackage{float} %for figure position

\setcounter{MaxMatrixCols}{20} %amsmath uses 10 array columns internally

\usepackage{multirow}

%\usepackage{subfig}
\usepackage[labelfont={bf},skip=15pt]{caption,subcaption} %,


%========================================
%                           MoVo                  
%========================================

% correct bad hyphenation here
\hyphenation{op-tical net-works semi-conduc-tor}

\newcommand{\shortname}{MoVo}
%
%\usepackage{subcaption} %\subfigure
%\usepackage[pdftex]{graphicx}
%\graphicspath{{figs/}}

%
\usepackage{pifont} %\cmark \xmarks
\newcommand{\cmark}{\ding{51}}%
\newcommand{\xmark}{\ding{55}}%

%\usepackage{lscape} %\landscape paper
%\usepackage{longtable}
%\usepackage{booktabs}
%\usepackage{amsmath}
%\usepackage{hyperref}

%\usepackage{tikz}  %\tikzpicture

%\usepackage[para,symbol*]{footmisc}
